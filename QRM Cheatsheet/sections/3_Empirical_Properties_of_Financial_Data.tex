\section{Empirical Properties of Financial Data}
\pink{3.1 Stylized facts of financial return series}
\begin{itemize}[leftmargin=*]
    \item \green{Stylized facts} are a collection of empirical observations and inferences drawn of such, which apply to many time series of risk-factor changes (e.g. log-returns of equities, indices, exchange rates, commodity prices).
    \item Stylized facts often apply to daily log-returns (also to intra-daily, weekly, monthly). Tick-by-tick (high-frequency) data have their own stylized facts (not discussed here) and annual return (low-frequency) data are more difficult to investigate (data sparsity; non-stationarity).
    \item Consider discrete-time risk-factor changes $X_{t}=Z_{t}-Z_{t-1}$, e.g. $Z_{t}=\log S_{t}$, in which case
$$
X_{t}=\log \left(S_{t} / S_{t-1}\right) \approx S_{t} / S_{t-1}-1=\left(S_{t}-S_{t-1}\right) / S_{t-1}
$$

the former is called \green{(log-)return}, the latter \green{(classical) return}.
\end{itemize}





\subsection*{Autocorrelation function (ACF)}
$\rho_{h}=\operatorname{corr}\left(X_{0}, X_{h}\right)$ for $h \in \mathbb{Z}$

Non-zero ACF at lag 1 would imply a tendency for a return to be followed by a return of equal sign; Not the case $\rightsquigarrow$ sign of the next return cannot be predicted.

iid data would imply $\rho_{X}(h)=\rho_{|X|}(h)=1_{\{h=0\}}$. Not the case $\rightsquigarrow$ $\left(X_{t}\right)$ is not a random walk





\subsection*{Testing univariate distributions}
\begin{itemize}[leftmargin=*]
    \item Let $x_{1}, \ldots, x_{n}$ be realizations of iid random variables $X_{1}, \ldots, X_{n}$ with cdf $F$
    \item \green{Empirical distribution function (edf)}
$$
\hat{F}_{n}(x):=\frac{1}{n} \sum_{i=1}^{n} 1_{\left\{x_{i} \leq x\right\}}
$$
    \item \green{Law of Large Numbers}:
$$
\text { For all } x \in \mathbb{R}, \quad \hat{F}_{n}(x) \rightarrow F(x) \quad \mathbb{P} \text {-almost surely }
$$
    \item \green{Glivenko-Cantelli Theorem}:
$$
\sup _{x \in \mathbb{R}}\left|\hat{F}_{n}(x)-F(x)\right| \rightarrow 0 \quad \mathbb{P} \text {-almost surely }
$$
\end{itemize}





\subsubsection*{Standard statistical tests for general cdf $F$}
For \red{general univariate} cdf $F$:
\begin{itemize}[leftmargin=*]
    \item \melon{Kolmogorov-Smirnov test}: $\quad T_{n}=\sup _{x \in \mathbb{R}}\left|\hat{F}_{n}(x)-F(x)\right|$
    \item \melon{Cramér-von Mises test}: $T_{n}=n \int_{\mathbb{R}}\left[\hat{F}_{n}(x)-F(x)\right]^{2} d F(x)$
    \item \melon{Anderson-Darling test}:
$$
T_{n}=n \int_{\mathbb{R}} \frac{\left[\hat{F}_{n}(x)-F(x)\right]^{2}}{F(x)(1-F(x))} d F(x)
$$
\end{itemize}
For \red{normal} $F \sim \N\left(\mu, \sigma^{2}\right)$:

\melon{Jarque-Bera test}: Let $\hat{\beta}_{n}$ and $\hat{\kappa}_{n}$ be sample versions of the 
(1) \green{skewness}: $\beta=\frac{\mathbb{E}(X-\mu)^{3}}{\sigma^{3}} \quad$ and 

(2) \green{kurtosis}: $\kappa=\frac{\mathbb{E}(X-\mu)^{4}}{\sigma^{4}}$


Then under the null-hypothesis, for large $n$,
$\frac{n}{6}\left(\hat{\beta}_{n}^{2}+\frac{\left(\hat{\kappa}_{n}-3\right)^{2}}{4}\right) \sim \chi_{2}^{2} \quad$

Note:
\begin{itemize}[leftmargin=*]
    \item Financial data is not symmetric, but neg-skewed.
    \item For normal dist, skewness = 0, kurtosis = 3
\end{itemize}





\subsubsection*{Graphical tests}
Denote by $x_{(1)} \leq \ldots \leq x_{(n)}$ the \green{ordered sample} and note that
$$
\hat{F}_{n}(x):=\frac{1}{n} \sum_{i=1}^{n} 1_{\left\{x_{i} \leq x\right\}}=\frac{1}{n} \sum_{i=1}^{n} 1_{\left\{x_{(i)} \leq x\right\}}
$$
\begin{itemize}[leftmargin=*]
    \item \green{P-P plot}
$$
\begin{aligned}
    &\left(p_{i}, F\left(x_{(i)}\right)\right), \quad i=1, \ldots, n, \\
    \text { where } \quad &p_{i}=\frac{i-1 / 2}{n} \approx \frac{i}{n} \approx \hat{F}_{n}\left(x_{(i)}\right)
\end{aligned}
$$

If $\hat{F}_{n} \approx F$, the points are close to the diagonal.
    \item \green{Q-Q plot}
$$
\begin{aligned}
    &\left(q\left(p_{i}\right), x_{(i)}\right), \quad i=1, \ldots, n, \\
    \quad \text { where } \quad &u \mapsto q(u) \text { is a quantile function of } F
\end{aligned}
$$

Again, if $\hat{F}_{n} \approx F$, the points are close to the diag.
    \item Typically, \red{tail differences are better visible in a Q-Q plot} than in a P-P plot. So, Q-Q plot are more widely used.
\end{itemize}



\subsection*{Q-Q Plot}
    \begin{itemize}[leftmargin=*]
    \item If $\hat{F}_{n} \approx N(0,1)$, points are close to the diagonal
    \item If $\hat{F}_{n} \approx N\left(\mu, \sigma^{2}\right)$, points are close to the line $y=\mu+\sigma x$
    \item S-shape hints at heavier tails than those of a normal distribution
    \item Daily returns typically have kurtosis $\kappa>3$
    
\green{leptokurtic}: narrower center, heavier tails than $N\left(\mu, \sigma^{2}\right)$, for which $\kappa=3$
    \item They typically have power-like tails
\end{itemize}




\subsection*{Longer time-interval return series}
\begin{itemize}[leftmargin=*]
    \item By going from daily to weekly, monthly, quarterly and yearly data, these effects become less pronounced (returns look more iid, less heavy-tailed)
    \item The (\red{non-overlapping)} $h$-period log-return at $t \in\{h, 2 h, \ldots, n h\}$ is
$$
\begin{aligned}
    X_{t}^{(h)}&=\log \left(\frac{S_{t}}{S_{t-h}}\right)=\log \left(\frac{S_{t}}{S_{t-1}} \frac{S_{t-1}}{S_{t-2}} \cdots \frac{S_{t-h+1}}{S_{t-h}}\right) \\
    &=\sum_{i=0}^{h-1} X_{t-i}
\end{aligned}
$$

A \green{Central Limit Theorem (CLT)} effect takes place (less heavy-tailed, less evidence of serial correlation)
    \item Problem: for larger $h$, less data is available
    \item Possible remedy: Consider \green{overlapping returns}
$
\left\{X_{t}^{(h)}: t \in\{h, h+k, h+2 k, \ldots, h+n k\}\right\} \text { for } 1 \leq k<h
$
$\leadsto$ more data, but \green{serially dependent} now
\end{itemize}







\subsection*{Summary: Stylized Facts about Univariate Financial Return Series}
\begin{enumerate}[label = (U\arabic*), leftmargin=*]
    \item Return series are not iid although they show little serial correlation
    \item Series of absolute or squared returns show profound serial correlation
    \item Conditional expected returns are close to zero
    \item Volatility (conditional standard deviation) appears to vary over time
    \item Extreme returns appear in clusters
    \item Return series are leptokurtic or heavy-tailed (power-like tail)
\end{enumerate}




\pink{3.2 Multivariate stylized facts}
\subsection*{Correlation between different assets}
Consider $d$-dimensional vectors of log-return data $X_{1}, \ldots, X_{n}$
\begin{itemize}[leftmargin=*]
    \item By (U1), the returns of $\operatorname{stock} A$ at times $t$ and $t+h$ show little correlation
    \item The same applies to the returns of $\operatorname{stock} A$ at time $t$ and stock $B$ at time $t+h$
    \item Returns of two different stocks on the same day may be correlated due to common underlying factors (\green{contemporaneous dependence})
    \item Contemporaneous correlations of returns vary over time (difficult to detect whether changes are continuous or constant within regimes)
fit different models for changing correlation, then make a formal comparison
    \item Periods of high/low volatility are typically common to more than one stock $\rightsquigarrow$ Returns of large magnitude of stock $A$ may be followed by returns of large magnitude of stocks $A$ and $B$
\end{itemize}



\subsection*{Tail Dependence}
\begin{itemize}[leftmargin=*]
    \item The normal distribution cannot replicate \melon{tail dependence}
    \item The $t_{3}$ distribution can produce \melon{joint large gains/losses} but in a \melon{symmetric way}.
\end{itemize}



\subsection*{Summary: Stylized Facts about Multivariate Financial Return Series}
\begin{enumerate}[label = (M\arabic*), leftmargin=*]
    \item Multivariate return series show little evidence of cross-correlation, except for contemporaneous returns (i.e. at the same time t)
    \item Multivariate series of absolute returns show profound cross-correlation
    \item Correlations between contemporaneous returns vary over time
    \item Extreme returns in one series often coincide with extreme returns in several other series (e.g. tail dependence)

\end{enumerate}