\section{Operational Risk}
\subsection*{Definition of Operational Risk}
The risk of loss resulting from inadequate or failed
internal processes, people and systems or from external
events.

This definition includes
\begin{itemize}[leftmargin=*]
    \item \green{Legal Risk} – financial loss that can result from lack
of awareness or misunderstanding of, ambiguity
in, or reckless indifference to, the way law and
regulation apply to your business, its relationships,
processes, products and services

but excludes
    \item \green{Strategic Risk} – loss arising from a poor strategic
business decision
    \item \green{Reputational Risk} – damage to an organization
through loss of its reputation or standing
\end{itemize}





\subsection*{Basel Op Risk categories}
\begin{enumerate}[label = (\arabic*), leftmargin=*]
    \item \green{Internal Fraud} – misappropriation of assets, tax
evasion, intentional mismarking of positions, bribery
    \item \green{External Fraud} – theft of information, hacking
damage, third-party theft and forgery
    \item \green{Employment Practices and Workplace Safety} –
discrimination, workers compensation
    \item \green{Clients, Products, and Business Practice} – market
manipulation, product defects
    \item \green{Damage to Physical Assets} – natural disasters,
terrorism, vandalism
    \item \green{Business Disruption and Systems Failures} – utility
disruptions, software failures, hardware failures
    \item \green{Execution, Delivery, and Process Management} –
data entry errors, accounting errors
\end{enumerate}






\pink{8.1 Basel Pillar 1 - Minimal Capital Requirements}
\subsection*{Regulatory Capital (RC)}
Banks must hold a regulatory minimum capital to absorb losses from Op Risk. Currently, there are \red{three approaches}.

\subsubsection*{Basic Indicator Approach (BIA)}
The Regulatory Capital under the BIA equals $15 \%$ of the average annual gross income over the previous three years where it was positive, i.e.,
$$
\mathrm{RC}_{t}^{\mathrm{BIA}}=15 \% \cdot\left(\sum_{k=1}^{3} \max \left\{\mathrm{GI}_{t-k}, 0\right\}\right) /\left(\sum_{k=1}^{3} \mathbb{I}_{\mathrm{GI}_{t-k}>0}\right)
$$
where $G l_{t-k}$ is the gross income in year $t-k$.
\begin{itemize}[leftmargin=*]
    \item For regional, non-complex firms
    \item Not risk-sensitive
\end{itemize}






\subsection*{The Standardized Approach (TSA)}
The TSA is like the BIA, but the calculation is performed separately for each business line with different weights, i.e.,
$$
\mathrm{RC}_{t}^{\mathrm{TSA}}=\frac{1}{3} \cdot \sum_{k=1}^{3} \max \left\{\sum_{b=1}^{8} \beta_{b} \mathrm{GI}_{t-k}^{b}, 0\right\}
$$
where $\mathrm{Gl}^{\mathrm{b}} \mathrm{t-k}$ is the gross income in year $\mathrm{t}-\mathrm{k}$ of business line $b$ and $\beta_{b}$ is its weight. The 8 business lines and their weights are (note the sum of weights is equal $1.2=8 \times 15 \%$):
$\begin{array}{ll}\text { Corporate finance } 18 \% & \text { Payment \& Settlement } 18 \% \\ \text { Trading \& Sales } 18 \% & \text { Agency Services } 15 \% \\ \text { Retail banking } 12 \% & \text { Asset management } 12 \% \\ \text { Commercial banking } 15 \% & \text { Retail brokerage } 12 \%\end{array}$
\begin{itemize}[leftmargin=*]
    \item Expected for most financial services firms
    \item Not risk-sensitive
\end{itemize}



\subsection*{Advanced Measurement Approach (AMA)}
The Regulatory Capital is equal to the \melon{Op Risk loss that is exceeded only once in 1000 years}, i.e., $\mathrm{VaR}_{0.999}(\mathrm{~L})$, where the random variable $\mathrm{L}$ is the annual Op Risk loss.

Common approach to model $L$, taken in large banks, is the Loss Distribution Approach (LDA)

Allows banks to use their internally generated risk estimates, based on extensive Supervisory Guidance:

(1) Operational Risk - Supervisory Guidelines for the Advanced Measurement Approaches, Basel Committee on Banking Supervision

(2) Supervisory Guidance for Data, Modeling, and Model Risk Management Under the Operational Risk Advanced Measurement Approaches, FED ... (and more)

\begin{itemize}[leftmargin=*]
    \item Is currently in place at many of the global and systemically important banks (e.g., UBS)
    \item Requires prior regulatory approval
    \item Involves complex statistical modelling, allows for flexibility
\end{itemize}





\subsection*{Loss Distribution Approach (LDA) within the AMA Framework}
\subsubsection*{Define Units of Measure (UoM)}
A UoM typically combines business lines and loss event types, e.g., Investment Bank and Fraud
\subsubsection*{Model the annual UoM loss as compound sum of Frequency and Severity}
For each $U \circ M \mathrm{u}, \mathrm{u} \in\{1,2, \ldots, \mathrm{U}\}$, annual loss $\mathrm{L}_{\mathrm{u}}=\sum_{\mathrm{k}=1}^{\mathrm{N}_{\mathrm{u}}} \mathrm{X}_{\mathrm{k}, \mathrm{u}}$, where $\left\{\mathrm{X}_{\mathrm{k}, \mathrm{u}}: \mathrm{k}=1,2, \ldots, \mathrm{N}_{\mathrm{u}}\right\}$ are i.i.d and independent from $\mathrm{N}_{\mathrm{u}}, \mathrm{N}_{\mathrm{u}}$ is the number of losses in UoM u per year (Frequency) and $\mathrm{X}_{\mathrm{k}, \mathrm{u}}$ is the amount of the $\mathrm{k}$-th loss in UoM $\mathrm{u}$ (Severity)
\subsubsection*{Aggregate the annual UoM losses into an annual loss distribution}
The annual Op Risk loss $\mathrm{L}$ is then $\mathrm{L}=\sum_{\mathrm{u}=1}^{\mathrm{U}} \mathrm{L}_{\mathrm{u}}$, where dependence structure of $\left(\mathrm{L}_{1}, \mathrm{~L}_{2}, \ldots, \mathrm{L}_{\mathrm{U}}\right)$ is given by Copula C
\subsubsection*{Very challenging modelling problem}
To estimate/justify: Segmentation into UoM / Frequency Distribution / Severity Distribution / Copula






\subsection*{The Four Data Elements for LDA Estimation}
1. ILD; 2. ELD; 3. SA; 4. BEICF

\melon{More objective but backward-looking}:

1. \green{Internal Operational Loss Event Data (ILD)}: Most relevant to the banks particular
case, and well known

2. \green{External Operational Loss Event Data (ELD)}:
\begin{itemize}[leftmargin=*]
    \item Data consortia, e.g., Operational Riskdata
eXchange Association (homogeneous
classification standards, data relevance)
    \item Publicly available data, e.g., media or
annual reports (reporting bias)
\end{itemize}



\melon{Forward-looking but more subjective}:

3. \green{Scenario Analysis (SA)}:
\begin{itemize}[leftmargin=*]
    \item Systematic process of obtaining expert opinions on the
likelihood and loss impact of plausible, high-severity
operational losses, typically developed through
workshops
    \item Expert biases (overconfidence, anchoring, …) and
subjectivity
\end{itemize}


3. \green{Business Environment and Internal Control Factors (BEICF)}:
\begin{itemize}[leftmargin=*]
    \item Indicators designed to provide a forward-looking
assessment of a banking organization's business risk
factors and internal control environment (impact of
discontinuing a line of business, a change in the
internal control environment, …)
    \item Might be used to adjust operational risk exposure
\end{itemize}






\subsection*{Modelling Options within the LDA Framework}
\begin{itemize}[leftmargin=*]
    \item Frequency: Poisson, Negative Binomial
    \item Severity: Log-Normal, Log-Gamma, Generalized Pareto, …
    \item Copula / Dependence
    \begin{itemize}[leftmargin=*]
        \item Dependence between annual losses, $\mathrm{L}_{\mathrm{u}}, \mathrm{u} \in\{1,2, \ldots, \mathrm{U}\}$, vs dependence on frequency / severity level
        \item Copulas: t, Clayton, Gumbel, Frank, ...
    \end{itemize}
    \item Use of the four data elements (ILD, ELD, SA, BEICF): 
    – Filtering ELD to remove non-relevant events; 
– Scaling ELD to account for differences in size or business activities; 
– Mixing data vs mixing distributions, e.g., fit distribution to ILD plus weighted ELD vs fit distributions to both ILD and ELD and mix the densities; 
– Benchmarking, e.g., compare ILD based main model with ELD based challenger model
– Build an own SA distribution vs SA based adjustments; 
– Bayesian approach: use SA distributions as prior and calculate posterior given ILD and ELD; 
– Parameter adjustments based on BEICF
\end{itemize}




\subsection*{Criticism of the AMA}
\begin{itemize}[leftmargin=*]
    \item Comparability of AMA minimum capital figures is questionable due to the full methodological freedom within the LDA
    \item How reliable are the quantitative estimates? 1-in-1000-year loss vs twenty years of ILD!
    
    \begin{itemize}[leftmargin=*]
    \item Limited availability of data / high confidence levels $\rightarrow$ uncertainty / instability in estimates
    \item Over-fitting and extrapolation challenges
    \end{itemize}
\end{itemize}




\subsection*{Standardized Measurement Approach (SMA)}
\begin{itemize}[leftmargin=*]
    \item A simpler and more comparable approach will be implemented effective 1 January 2023*, see Section Minimum capital
requirements for operational risk in Basel III: Finalising post-crisis reforms from the Basel Committee on Banking Supervision
    \item The SMA combines the Business Indicator, a simple financial statement proxy of operational risk exposure, with bankspecific
Internal Loss Multiplier (based on internal operational loss data) to provide some incentive for banks to improve
their operational risk management
    \item Nevertheless, comparability of capital charges remains a concern because
    \begin{itemize}[leftmargin=*]
    \item the collection of operational loss data is still determined by individual bank rules (no common standard),
    \item national regulators may grant exclusion of (parts of) the loss history from the calculation (due to backward-looking nature of the SMA).
    \end{itemize}
\end{itemize}






\pink{8.2 Basel Pillar 2 -  Supervisory Review of Capital Adequacy}
\subsection*{Supervisory Review of Capital Adequacy}
\begin{itemize}[leftmargin=*]
    \item In the Internal Capital Adequacy Assessment Process, a
bank needs to assess (among other things) whether it
considers its capital adequate to cover the level and
nature of the risks to which it is exposed
    \item This includes the risk types from Pillar 1 (Credit Risk,
Market Risk and Op Risk), but extends to every possible
risk type and their aggregation / diversification
\end{itemize}


\subsection*{Stress Testing}
Stress Testing: analysis to determine whether there is enough capital to withstand the impact of an unfavorable economic scenario, including the causality
chain by which losses would arise if the scenario were to unfold





\subsection*{Modelling Op Risk Stress Loss}
1. Leverage the Loss Distribution Approach by choosing a particular quantile to represent the loss under stress
\begin{itemize}[leftmargin=*]
    \item Choice of the quantile is difficult to substantiate, which has led all CCAR banks to move away from this approach.
\end{itemize}
2. Regressions and case studies / scenarios analysis
\begin{itemize}[leftmargin=*]
    \item Quarterly operational losses may be linked to macroeconomic risk factor in certain cases, e.g., losses in “Execution, delivery, and process
management” or “External fraud” to transaction volumes, real estate/house prices, GDP, and delinquency or unemployment rates.
    \item A possible modelling approach is to use generalized linear models with compound distributions (e.g., Poisson-Gamma) and log-link
functions. The stability of the model and the lack of sufficient predictive power can be a challenge.
    \item A pool of candidate models is usually identified and the final model is selected by subject-matter experts and business intuition. If no
economically meaningful model is found, fallback estimates may be used, e.g., historical average losses.
    \item To capture the possibility of rare (or not yet experienced) events, to include risk controls and mitigation efforts, to include a more
“forward-looking” view, to evaluate the vulnerabilities of the bank identified during a risk identification exercise and to overcome some of
the challenges when building a quantitative model, case study/scenario analysis is conducted by experts in workshops and the
corresponding loss estimates (or loss estimate refinements) are included in the projections.
\end{itemize}




\subsection*{Economic Capital Calculation}
\begin{itemize}[leftmargin=*]
    \item Economic Capital: amount of capital required to ensure solvency over a year with a pre-specified probability (e.g., 95\% or 99.9\%).
    \item A common approach to generate the annual loss distribution is to model risk types via their marginal loss distributions and then to
aggregate them using copulas, see picture. For the Op Risk marginal distribution, the LDA can be used again (if there is already AMA LDA
model).
    \item Another approach is to simulate many economic scenarios (risk drivers) consistently, and then to use methods like the ones used in Stress
Testing to calculate the annual loss for each scenario.
\end{itemize}







\pink{8.3 Monitoring \& Surveillance of Op Risks - Compliance Models}
\subsection*{Compliance Models – Landscape}
\begin{itemize}[leftmargin=*]
    \item \green{Adverse Media Screening}: Identify financial crime relevant news articles from various media sources concerning UBS clients or prospects
    \item \green{Sanctions Screening}: Identify references to sanctioned
entities, individuals or regions
within payment messages in
order to prevent the transaction
    \item \green{AML Client Risk Rating}: Produce AML risk ratings which
drive the frequency/depth of
periodic client reviews and the
level of alerting thresholds in
downstream transaction
monitoring systems
    \item \green{AML Transaction Monitoring}: Detect suspicious client
transactions relating to money
laundering (e.g., changes in
transaction behavior,
transactions involving high risk
jurisdictions, flow through)
    \item \green{Communication Monitoring}: Detect potential compliance
breaches in electronic
messages (chats, e-mails, etc.)
and audio communication
(landline, mobile, Skype) of
targeted UBS employees
    \item \green{Trade Surveillance}: Detect potential cases of
market misconduct (e.g.,
insider trading, front running
and trades away from the
market price)
\end{itemize}





\subsection*{Compliance Models – Characteristics and Testing}
\begin{itemize}[leftmargin=*]
    \item Use
    \begin{itemize}[leftmargin=*]
        \item Many models monitor key
operational risks (Financial
Crime, Market Conduct)
        \item Alerts go through an expert
review process and might
ultimately lead to a
regulatory filing
    \end{itemize}
    
    \item Input data
    \begin{itemize}[leftmargin=*]
    \item Large amounts of data
(trades, orders, text, audio,
transactions, payments,
client data), typically
sourced from core systems
    \item Processing is usually
automated
    \end{itemize}
    
    \item Methodology
    \begin{itemize}[leftmargin=*]
    \item Monthly, daily or event
based execution of the
alerting logic
    \item Many submodels based on
rules with many tunable
parameters, statistical
anomaly detection and/or
Machine Learning
    \end{itemize}
    
    
    
    
    \item Implementation
    \begin{itemize}[leftmargin=*]
    \item Inhouse built systems as
well as on- and off-premise
vendor solutions
    \item Implementation under
resposibility of the IT
department
    \end{itemize}
    
    
    \item \red{Key model risk are
false negatives (Type II error)}: The model does not produce an alert when it should have (false alerts “only” lead to
extra effort and are well controlled through alert review)

    \item \green{Key testing / controls}: 
        \begin{itemize}[leftmargin=*]
\item Regular reviews of non-alerting cases / Below-the-Line testing
\item Regular testing with synthetic data
\item Regular coverage assessments (regulatory/internal requirements)
    \end{itemize}
\end{itemize}