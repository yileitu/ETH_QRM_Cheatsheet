\section{Introduction}
\subsection*{Model Uncertainty}
Model uncertainty refers to the uncertainty about the accuracy of a model. It results
from imprecise and idealized assumptions, which, to some degree, have to be made in every
modeling framework.

\subsection*{Different Types of Risk}
Market risk; Credit risk; Liquidity risk [including (1)Market liquidity risk, (2)Funding liquidity risk]; Operational risk; Underwriting risk; Model risk (an example of Knightian uncertainty)


\subsection*{What does managing risks involve?}
\begin{itemize}[leftmargin=*]
    \item Determine enough buffer capital to absorb losses (for regulatory purposes and economic capital purposes).
    \item Making sure portfolios are well diversified.
    \item Optimizing portfolios according to risk-return considerations.
\end{itemize}



\subsection*{Recent developments and concerns}
High frequency trading; Algorithmic trading; Commodities trading; Systemic risk



\subsection*{Basel Accord}
\subsubsection*{The first Basel Accord (Basel I), 1988}
\begin{itemize}[leftmargin=*]
    \item Only addressed credit risk.
    \item Fairly coarse measurement of risk. Loans were divided into 3 categories only, counterparties
being governments, regulated banks and others.
    \item Risk weighting identical for all corporate borrowers, independent of their credit rating.
    \item Unsatisfactory treatment of derivatives.
    \item Banks pushed to be allowed to use netting
    \item Amendment to Basel 1 in 1996: (1) Standardized model for market risk and internal VaR-based models for more sophisticated banks.
(2) Coarseness problem for credit risk remained.
\end{itemize}


\subsubsection*{The second Basel Accord (Basel II), 2004}
\green{Three pillar concept}:
\begin{enumerate}[label = Pillar \arabic*, leftmargin=*]
    \item \melon{Minimal capital charge}:
Requirements for the calculation of the regulatory capital to ensure that a bank holds sufficient
capital for its credit risk in the banking book, its market risk in the trading book and
operational risk (which was added as a new class of risk)
    \item \melon{Supervisory review process}:
Local regulators review the checks and balances put in place for capital adequacy assessments,
ensure that banks have adequate regulatory capital and perform stress tests of a bank’s capital
adequacy
    \item \melon{Market discipline}: 
Banks are required to make risk management processes more transparent
\end{enumerate}





\subsubsection*{Basel 2.5, 2009}
\begin{itemize}[leftmargin=*]
    \item CDOs provided opportunities for regulatory arbitrage (transferring credit risk from the capital-intensive banking book to the less-capitalized trading book)
    \item The aim of Basel 2.5 was to address the build up of risk in the trading book
    \item It included (1) \melon{stressed VaR} calculations of positions in the trading book; (2)
\melon{incremental risk charge} due to possible defaults and rating changes;
(3) Exposure to \melon{securitizations} in the trading book was subjected to new capital charges
\end{itemize}



\subsubsection*{The third Basel Accord (Basel III), 2010}
Intends to increase bank liquidity and decrease bank leverage.

Five extensions:
\begin{enumerate}[label = (\arabic*), leftmargin=*]
    \item It increases the quality and amount of capital by changing the definition of key capital ratios and allowing countercyclical adjustments to these ratios in crises
    \item It strengthens the framework for counterparty credit risk in derivatives trading with incentives to
use central counterparties (exchanges)
    \item It introduces a leverage ratio to prevent excessive leverage (a way to multiply gains/losses by
buying more of an asset with borrowed capital)
    \item It introduces various ratios that ensure that banks have sufficient funding liquidity
    \item It forces systemically important financial institutions (SIFIs) to hold even more risk capital
\end{enumerate}


\subsubsection*{Basel IV (anticipated)}
\begin{itemize}[leftmargin=*]
    \item Would require more stringent capital requirements
    \item Emphasizes simpler or standardized models in place of bank internal models
    \item Requires more detailed disclosure of reserves and other financial statistics
\end{itemize}



\subsection*{From Solvency I to II}
\begin{itemize}[leftmargin=*]
    \item Solvency I: Rather coarse rules-based framework calling for companies to have a minimum guarantee fund. Simple robust system, easy to understand, inexpensive to monitor. However, it is mainly volume based and not explicitly risk based.
    \item Goals of Solvency II: strengthen the capital adequacy by reducing the possibilities of consumer
loss or market disruption in insurance (policyholder protection and financial stability motives)
    \item Solvency II is also based on a three-pillar system
    
\green{Pillar 1} quantification of regulatory capital

\green{Pillar 2} governance and supervision

\green{Pillar 3} disclosure of information to the public

    \item Under Pillar 1, a company calculates its \green{solvency capital requirement (SCR)} = minimal
amount of capital ensuring that the probability of insolvency over a one-year period is no more
than 0.5\%. If this level of capital is not reached it will likely result in regulatory intervention
and require remedial action.
    \item The firm must also calculate \green{minimum capital requirement (MCR)} = minimum capital to cover
it’s risks. If an insurer violates the MCR constraint, it will be prohibited from writing any further business.
    \item For calculating capital requirements, a \green{standard formula} or an \green{internal model} may be used. Either way, a total balance sheet approach is taken (all risks and their interactions are considered)
The insurer should have own funds (surplus of assets over liabilities) exceeding both SCR and MCR
    \item Under Pillar 2, the company must demonstrate that it has a RM system in place and that this system is integrated into decision making processes
    \item An internal model must pass the “use test": It must be an integral part of the RM system and be actively used in the running of the firm. Moreover, a firm must undertake an \green{ORSA (own risk and solvency assessment)}
    \item An \green{internal model} often takes the form of a so-called \green{economic scenario generator (ESG)} in which risk-factor scenarios for a one-year period are randomly generated and applied to determine the \green{SCR}.
    
\end{itemize}



\subsection*{ORSA (Own risk and solvency assessment)}
ORSA = Entirety of processes and procedures to identify, assess, monitor, manage, and report
short and long term risks a (re)insurance company may face and to determine the own funds
necessary to ensure the company’s solvency at all times.

ORSA (Pillar 2) is different from capital calculations (Pillar 1).
\begin{itemize}[leftmargin=*]
    \item ORSA refers to a process (and not just an exercise in regulatory compliance)
    \item Each firm’s ORSA is its own process and likely to be unique (not bound by a common set of rules such as the standard-formula approach in Pillar 1)
    \item ORSA goes beyond the one-year time horizon (which is a limitation of Pillar 1); e.g. for life insurance
\end{itemize}



\subsection*{Benefits \& Criticism of regulatory frameworks}
\melon{Benefits} of regulation: Customer protection, responsible corporate governance, fair and comparable accounting rules, transparent information on risk, capital and solvency for shareholders etc.

\melon{Criticism}: 
\begin{itemize}[leftmargin=*]
    \item \green{Costs and complexity} for setting up and maintaining a sound risk management system compliant with present regulations (in the UK, Solvency II compliance costs at least 3 billion pounds). Regulation becomes more and more complex.
    \item \green{Endogenous risk} Regulation may amplify shocks. It can lead to risk-management herding (institutions all run for the same exit by following the same (perhaps VaR-based) rules in times of crisis and thus further destabilize the whole system).
    \item \green{Market consistent valuation} (at the core of the Basel rules for the trading book and Solvency II) implies that capital requirements are closely coupled to volatile financial markets.
\end{itemize}







\subsection*{Why manage financial risk?}
\subsubsection*{Societal view}
\begin{itemize}[leftmargin=*]
    \item Society relies on the stability of the banking and insurance system. The regulatory process, from which Basel II and Solvency II resulted, was motivated by the desire to prevent insolvency of individual institutions and thus protect customers (\green{microprudential perspective})
    \item Since the 2007–2009 crisis, the reduction of systemic risk has become an important secondary focus (\green{macroprudential perspective})
    \item The interests of society are served by enforcing the discipline of risk management in financial firms, through the use of regulation. Better risk management can reduce the risk of company failure and protect customers and policyholders. However, regulation must be designed with care and should not promote herding, procyclical behaviour or other forms of endogenous risk that could result in a systemic crisis. Individual firms need to be allowed to fail on occasion.
\end{itemize}

\subsubsection*{Shareholders’ view}
\begin{itemize}[leftmargin=*]
    \item While individual investors are typically risk averse and should therefore manage the risk in their portfolios, it is not immediately clear that risk management at the corporate level (e.g. holding a certain amount of risk capital or hedging a foreign currency exposure) increases the value of a corporation and thus enhances shareholder value.
    \item Theoretically, if investors have access to perfect capital markets, they can incorporate RM via their own trading and diversification.
    \item \green{The Modigliani–Miller theorem}, which marks the beginning of modern corporate finance theory, states that, in an ideal world without taxes, bankruptcy costs and informational asymmetries, and with frictionless and arbitrage-free capital markets, the financial structure of a firm (thus its RM decisions) is irrelevant for a firm’s value.
\end{itemize}


\subsection*{Reasons for corporate RM}
\begin{itemize}[leftmargin=*]
    \item RM can reduce taxes.
    \item RM can be beneficial, since a company may have better access to capital markets than individual investors.
    \item RM can increase the firm value in the presence of bankruptcy costs (liquidation costs or litigation costs), as it reduces the likelihood of bankruptcy.
    \item RM can reduce the impact of costly external financing.
\end{itemize}